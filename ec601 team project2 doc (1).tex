%% bare_jrnl_transmag.tex
%% V1.4b
%% 2015/08/26
%% by Michael Shell
%% see http://www.michaelshell.org/
%% for current contact information.
%%
%% This is a skeleton file demonstrating the use of IEEEtran.cls
%% (requires IEEEtran.cls version 1.8b or later) with an IEEE 
%% Transactions on Magnetics journal paper.
%%
%% Support sites:
%% http://www.michaelshell.org/tex/ieeetran/
%% http://www.ctan.org/pkg/ieeetran
%% and
%% http://www.ieee.org/

%%*************************************************************************
%% Legal Notice:
%% This code is offered as-is without any warranty either expressed or
%% implied; without even the implied warranty of MERCHANTABILITY or
%% FITNESS FOR A PARTICULAR PURPOSE! 
%% User assumes all risk.
%% In no event shall the IEEE or any contributor to this code be liable for
%% any damages or losses, including, but not limited to, incidental,
%% consequential, or any other damages, resulting from the use or misuse
%% of any information contained here.
%%
%% All comments are the opinions of their respective authors and are not
%% necessarily endorsed by the IEEE.
%%
%% This work is distributed under the LaTeX Project Public License (LPPL)
%% ( http://www.latex-project.org/ ) version 1.3, and may be freely used,
%% distributed and modified. A copy of the LPPL, version 1.3, is included
%% in the base LaTeX documentation of all distributions of LaTeX released
%% 2003/12/01 or later.
%% Retain all contribution notices and credits.
%% ** Modified files should be clearly indicated as such, including  **
%% ** renaming them and changing author support contact information. **
%%*************************************************************************


% *** Authors should verify (and, if needed, correct) their LaTeX system  ***
% *** with the testflow diagnostic prior to trusting their LaTeX platform ***
% *** with production work. The IEEE's font choices and paper sizes can   ***
% *** trigger bugs that do not appear when using other class files.       ***                          ***
% The testflow support page is at:
% http://www.michaelshell.org/tex/testflow/



\documentclass[journal,transmag]{IEEEtran}
%
% If IEEEtran.cls has not been installed into the LaTeX system files,
% manually specify the path to it like:
% \documentclass[journal]{../sty/IEEEtran}





% Some very useful LaTeX packages include:
% (uncomment the ones you want to load)


% *** MISC UTILITY PACKAGES ***
%
%\usepackage{ifpdf}
% Heiko Oberdiek's ifpdf.sty is very useful if you need conditional
% compilation based on whether the output is pdf or dvi.
% usage:
% \ifpdf
%   % pdf code
% \else
%   % dvi code
% \fi
% The latest version of ifpdf.sty can be obtained from:
% http://www.ctan.org/pkg/ifpdf
% Also, note that IEEEtran.cls V1.7 and later provides a builtin
% \ifCLASSINFOpdf conditional that works the same way.
% When switching from latex to pdflatex and vice-versa, the compiler may
% have to be run twice to clear warning/error messages.






% *** CITATION PACKAGES ***
%
%\usepackage{cite}
% cite.sty was written by Donald Arseneau
% V1.6 and later of IEEEtran pre-defines the format of the cite.sty package
% \cite{} output to follow that of the IEEE. Loading the cite package will
% result in citation numbers being automatically sorted and properly
% "compressed/ranged". e.g., [1], [9], [2], [7], [5], [6] without using
% cite.sty will become [1], [2], [5]--[7], [9] using cite.sty. cite.sty's
% \cite will automatically add leading space, if needed. Use cite.sty's
% noadjust option (cite.sty V3.8 and later) if you want to turn this off
% such as if a citation ever needs to be enclosed in parenthesis.
% cite.sty is already installed on most LaTeX systems. Be sure and use
% version 5.0 (2009-03-20) and later if using hyperref.sty.
% The latest version can be obtained at:
% http://www.ctan.org/pkg/cite
% The documentation is contained in the cite.sty file itself.






% *** GRAPHICS RELATED PACKAGES ***
%
\ifCLASSINFOpdf
  % \usepackage[pdftex]{graphicx}
  % declare the path(s) where your graphic files are
  % \graphicspath{{../pdf/}{../jpeg/}}
  % and their extensions so you won't have to specify these with
  % every instance of \includegraphics
  % \DeclareGraphicsExtensions{.pdf,.jpeg,.png}
\else
  % or other class option (dvipsone, dvipdf, if not using dvips). graphicx
  % will default to the driver specified in the system graphics.cfg if no
  % driver is specified.
  % \usepackage[dvips]{graphicx}
  % declare the path(s) where your graphic files are
  % \graphicspath{{../eps/}}
  % and their extensions so you won't have to specify these with
  % every instance of \includegraphics
  % \DeclareGraphicsExtensions{.eps}
\fi
% graphicx was written by David Carlisle and Sebastian Rahtz. It is
% required if you want graphics, photos, etc. graphicx.sty is already
% installed on most LaTeX systems. The latest version and documentation
% can be obtained at: 
% http://www.ctan.org/pkg/graphicx
% Another good source of documentation is "Using Imported Graphics in
% LaTeX2e" by Keith Reckdahl which can be found at:
% http://www.ctan.org/pkg/epslatex
%
% latex, and pdflatex in dvi mode, support graphics in encapsulated
% postscript (.eps) format. pdflatex in pdf mode supports graphics
% in .pdf, .jpeg, .png and .mps (metapost) formats. Users should ensure
% that all non-photo figures use a vector format (.eps, .pdf, .mps) and
% not a bitmapped formats (.jpeg, .png). The IEEE frowns on bitmapped formats
% which can result in "jaggedy"/blurry rendering of lines and letters as
% well as large increases in file sizes.
%
% You can find documentation about the pdfTeX application at:
% http://www.tug.org/applications/pdftex




% *** MATH PACKAGES ***
%
%\usepackage{amsmath}
% A popular package from the American Mathematical Society that provides
% many useful and powerful commands for dealing with mathematics.
%
% Note that the amsmath package sets \interdisplaylinepenalty to 10000
% thus preventing page breaks from occurring within multiline equations. Use:
%\interdisplaylinepenalty=2500
% after loading amsmath to restore such page breaks as IEEEtran.cls normally
% does. amsmath.sty is already installed on most LaTeX systems. The latest
% version and documentation can be obtained at:
% http://www.ctan.org/pkg/amsmath





% *** SPECIALIZED LIST PACKAGES ***
%
%\usepackage{algorithmic}
% algorithmic.sty was written by Peter Williams and Rogerio Brito.
% This package provides an algorithmic environment fo describing algorithms.
% You can use the algorithmic environment in-text or within a figure
% environment to provide for a floating algorithm. Do NOT use the algorithm
% floating environment provided by algorithm.sty (by the same authors) or
% algorithm2e.sty (by Christophe Fiorio) as the IEEE does not use dedicated
% algorithm float types and packages that provide these will not provide
% correct IEEE style captions. The latest version and documentation of
% algorithmic.sty can be obtained at:
% http://www.ctan.org/pkg/algorithms
% Also of interest may be the (relatively newer and more customizable)
% algorithmicx.sty package by Szasz Janos:
% http://www.ctan.org/pkg/algorithmicx




% *** ALIGNMENT PACKAGES ***
%
%\usepackage{array}
% Frank Mittelbach's and David Carlisle's array.sty patches and improves
% the standard LaTeX2e array and tabular environments to provide better
% appearance and additional user controls. As the default LaTeX2e table
% generation code is lacking to the point of almost being broken with
% respect to the quality of the end results, all users are strongly
% advised to use an enhanced (at the very least that provided by array.sty)
% set of table tools. array.sty is already installed on most systems. The
% latest version and documentation can be obtained at:
% http://www.ctan.org/pkg/array


% IEEEtran contains the IEEEeqnarray family of commands that can be used to
% generate multiline equations as well as matrices, tables, etc., of high
% quality.




% *** SUBFIGURE PACKAGES ***
%\ifCLASSOPTIONcompsoc
%  \usepackage[caption=false,font=normalsize,labelfont=sf,textfont=sf]{subfig}
%\else
%  \usepackage[caption=false,font=footnotesize]{subfig}
%\fi
% subfig.sty, written by Steven Douglas Cochran, is the modern replacement
% for subfigure.sty, the latter of which is no longer maintained and is
% incompatible with some LaTeX packages including fixltx2e. However,
% subfig.sty requires and automatically loads Axel Sommerfeldt's caption.sty
% which will override IEEEtran.cls' handling of captions and this will result
% in non-IEEE style figure/table captions. To prevent this problem, be sure
% and invoke subfig.sty's "caption=false" package option (available since
% subfig.sty version 1.3, 2005/06/28) as this is will preserve IEEEtran.cls
% handling of captions.
% Note that the Computer Society format requires a larger sans serif font
% than the serif footnote size font used in traditional IEEE formatting
% and thus the need to invoke different subfig.sty package options depending
% on whether compsoc mode has been enabled.
%
% The latest version and documentation of subfig.sty can be obtained at:
% http://www.ctan.org/pkg/subfig



% *** FLOAT PACKAGES ***
%
%\usepackage{fixltx2e}
% fixltx2e, the successor to the earlier fix2col.sty, was written by
% Frank Mittelbach and David Carlisle. This package corrects a few problems
% in the LaTeX2e kernel, the most notable of which is that in current
% LaTeX2e releases, the ordering of single and double column floats is not
% guaranteed to be preserved. Thus, an unpatched LaTeX2e can allow a
% single column figure to be placed prior to an earlier double column
% figure.
% Be aware that LaTeX2e kernels dated 2015 and later have fixltx2e.sty's
% corrections already built into the system in which case a warning will
% be issued if an attempt is made to load fixltx2e.sty as it is no longer
% needed.
% The latest version and documentation can be found at:
% http://www.ctan.org/pkg/fixltx2e


%\usepackage{stfloats}
% stfloats.sty was written by Sigitas Tolusis. This package gives LaTeX2e
% the ability to do double column floats at the bottom of the page as well
% as the top. (e.g., "\begin{figure*}[!b]" is not normally possible in
% LaTeX2e). It also provides a command:
%\fnbelowfloat
% to enable the placement of footnotes below bottom floats (the standard
% LaTeX2e kernel puts them above bottom floats). This is an invasive package
% which rewrites many portions of the LaTeX2e float routines. It may not work
% with other packages that modify the LaTeX2e float routines. The latest
% version and documentation can be obtained at:
% http://www.ctan.org/pkg/stfloats
% Do not use the stfloats baselinefloat ability as the IEEE does not allow
% \baselineskip to stretch. Authors submitting work to the IEEE should note
% that the IEEE rarely uses double column equations and that authors should try
% to avoid such use. Do not be tempted to use the cuted.sty or midfloat.sty
% packages (also by Sigitas Tolusis) as the IEEE does not format its papers in
% such ways.
% Do not attempt to use stfloats with fixltx2e as they are incompatible.
% Instead, use Morten Hogholm'a dblfloatfix which combines the features
% of both fixltx2e and stfloats:
%
% \usepackage{dblfloatfix}
% The latest version can be found at:
% http://www.ctan.org/pkg/dblfloatfix




%\ifCLASSOPTIONcaptionsoff
%  \usepackage[nomarkers]{endfloat}
% \let\MYoriglatexcaption\caption
% \renewcommand{\caption}[2][\relax]{\MYoriglatexcaption[#2]{#2}}
%\fi
% endfloat.sty was written by James Darrell McCauley, Jeff Goldberg and 
% Axel Sommerfeldt. This package may be useful when used in conjunction with 
% IEEEtran.cls'  captionsoff option. Some IEEE journals/societies require that
% submissions have lists of figures/tables at the end of the paper and that
% figures/tables without any captions are placed on a page by themselves at
% the end of the document. If needed, the draftcls IEEEtran class option or
% \CLASSINPUTbaselinestretch interface can be used to increase the line
% spacing as well. Be sure and use the nomarkers option of endfloat to
% prevent endfloat from "marking" where the figures would have been placed
% in the text. The two hack lines of code above are a slight modification of
% that suggested by in the endfloat docs (section 8.4.1) to ensure that
% the full captions always appear in the list of figures/tables - even if
% the user used the short optional argument of \caption[]{}.
% IEEE papers do not typically make use of \caption[]'s optional argument,
% so this should not be an issue. A similar trick can be used to disable
% captions of packages such as subfig.sty that lack options to turn off
% the subcaptions:
% For subfig.sty:
% \let\MYorigsubfloat\subfloat
% \renewcommand{\subfloat}[2][\relax]{\MYorigsubfloat[]{#2}}
% However, the above trick will not work if both optional arguments of
% the \subfloat command are used. Furthermore, there needs to be a
% description of each subfigure *somewhere* and endfloat does not add
% subfigure captions to its list of figures. Thus, the best approach is to
% avoid the use of subfigure captions (many IEEE journals avoid them anyway)
% and instead reference/explain all the subfigures within the main caption.
% The latest version of endfloat.sty and its documentation can obtained at:
% http://www.ctan.org/pkg/endfloat
%
% The IEEEtran \ifCLASSOPTIONcaptionsoff conditional can also be used
% later in the document, say, to conditionally put the References on a 
% page by themselves.




% *** PDF, URL AND HYPERLINK PACKAGES ***
%
%\usepackage{url}
% url.sty was written by Donald Arseneau. It provides better support for
% handling and breaking URLs. url.sty is already installed on most LaTeX
% systems. The latest version and documentation can be obtained at:
% http://www.ctan.org/pkg/url
% Basically, \url{my_url_here}.




% *** Do not adjust lengths that control margins, column widths, etc. ***
% *** Do not use packages that alter fonts (such as pslatex).         ***
% There should be no need to do such things with IEEEtran.cls V1.6 and later.
% (Unless specifically asked to do so by the journal or conference you plan
% to submit to, of course. )


% correct bad hyphenation here
\hyphenation{op-tical net-works semi-conduc-tor}


\begin{document}
%
% paper title
% Titles are generally capitalized except for words such as a, an, and, as,
% at, but, by, for, in, nor, of, on, or, the, to and up, which are usually
% not capitalized unless they are the first or last word of the title.
% Linebreaks \\ can be used within to get better formatting as desired.
% Do not put math or special symbols in the title.
\title{LLM for Industrial Design}



% author names and affiliations
% transmag papers use the long conference author name format.

\author{\IEEEauthorblockN{Haozhe Pang\IEEEauthorrefmark{1},
Zhengyuan Li\IEEEauthorrefmark{1},
Jiaqi Wu\IEEEauthorrefmark{1}}
\IEEEauthorblockA{\IEEEauthorrefmark{1}Department of Electrical and Computer Engineering,
Boston University, Boston, MA 02215 USA}
\thanks{Guidence from Professor Osama Alshaykh\IEEEauthorrefmark{1}}}



% The paper headers
\markboth{601 Team Project2}%
{Shell \MakeLowercase{\textit{et al.}}: Bare Demo of IEEEtran.cls for IEEE Transactions on Magnetics Journals}
% The only time the second header will appear is for the odd numbered pages
% after the title page when using the twoside option.
% 
% *** Note that you probably will NOT want to include the author's ***
% *** name in the headers of peer review papers.                   ***
% You can use \ifCLASSOPTIONpeerreview for conditional compilation here if
% you desire.




% If you want to put a publisher's ID mark on the page you can do it like
% this:
%\IEEEpubid{0000--0000/00\$00.00~\copyright~2015 IEEE}
% Remember, if you use this you must call \IEEEpubidadjcol in the second
% column for its text to clear the IEEEpubid mark.



% use for special paper notices
%\IEEEspecialpapernotice{(Invited Paper)}


% for Transactions on Magnetics papers, we must declare the abstract and
% index terms PRIOR to the title within the \IEEEtitleabstractindextext
% IEEEtran command as these need to go into the title area created by
% \maketitle.
% As a general rule, do not put math, special symbols or citations
% in the abstract or keywords.
\IEEEtitleabstractindextext{%
\begin{abstract}
This paper explores the transformative impact of Large Language Models (LLMs) on both product development and customer engagement within the tech industry. By harnessing advanced natural language processing capabilities, LLMs facilitate a deeper understanding of user feedback and enable more informed decision-making across various roles, including developers, designers, product managers, quality assurance professionals, and marketing teams. We illustrate how LLMs analyze vast amounts of feedback to prioritize development tasks, enhance user interface designs, streamline product management processes, and improve marketing strategies. This multidisciplinary approach not only leads to more user-centered products but also optimizes the development cycle by integrating real user interactions and feedback. Additionally, the paper details the creation of a feedback system MVP that employs LLM analysis for actionable insights, demonstrating its application in practical settings. Through extensive literature reviews and user stories, we underscore the significance of LLMs in advancing product innovation and enriching customer interactions, asserting their indispensable role in the evolving landscape of the tech industry.
\end{abstract}

% Note that keywords are not normally used for peerreview papers.
\begin{IEEEkeywords}
Large Language Models, User Feedback, Product Development, Customer Engagement
\end{IEEEkeywords}}



% make the title area
\maketitle


% To allow for easy dual compilation without having to reenter the
% abstract/keywords data, the \IEEEtitleabstractindextext text will
% not be used in maketitle, but will appear (i.e., to be "transported")
% here as \IEEEdisplaynontitleabstractindextext when the compsoc 
% or transmag modes are not selected <OR> if conference mode is selected 
% - because all conference papers position the abstract like regular
% papers do.
\IEEEdisplaynontitleabstractindextext
% \IEEEdisplaynontitleabstractindextext has no effect when using
% compsoc or transmag under a non-conference mode.







% For peer review papers, you can put extra information on the cover
% page as needed:
% \ifCLASSOPTIONpeerreview
% \begin{center} \bfseries EDICS Category: 3-BBND \end{center}
% \fi
%
% For peerreview papers, this IEEEtran command inserts a page break and
% creates the second title. It will be ignored for other modes.
\IEEEpeerreviewmaketitle




\section{Definition of the project}
Using an LLM (Large Language Model) to modify an application or product design based on feedback from a demo involves leveraging the advanced natural language processing capabilities of the model to analyze user feedback. This process typically includes parsing the feedback to extract key insights, suggestions, and issues raised by users. The LLM can then generate actionable recommendations for design improvements or adjustments. These recommendations are used by developers or designers to refine the application or product, enhancing its usability, functionality, or appeal based on real user interactions and experiences. This approach helps in creating a more user-centric product by integrating direct feedback into the development cycle through the use of sophisticated AI-driven analysis.

\section{Target users}
We give the following multidisciplinary approach to ensure that various aspects of product development benefit from user feedback, leading to more effective and user-centered products.
\textbf{Developers and Designers:} Individuals who are actively involved in the development and design of applications and products. They utilize the insights generated by the LLM to make informed decisions about changes and improvements.

\textbf{Product Managers:} Professionals responsible for overseeing the development lifecycle of a product. They leverage the LLM’s analysis to align product features with market needs and user expectations.

\textbf{Quality Assurance Professionals:} Specialists who focus on ensuring the product meets certain standards and functions as intended. They use feedback analysis to identify and address potential issues before the product reaches the market.

\textbf{User Experience (UX) Designers:} Experts in creating optimal user interaction with the product. They rely on detailed feedback analysis to enhance the usability and aesthetic appeal of the application or product.

\textbf{Marketing Teams:} Groups tasked with promoting the product. They can use insights from the feedback to better understand customer needs and tailor marketing strategies accordingly.



\section{user stories}
In the rapidly evolving tech industry, the integration of Large Language Models (LLMs) is promising to enhance both customer engagement and product development processes. These advanced models are increasingly relied upon to extract valuable insights from vast amounts of data, helping teams across various functions make more informed decisions and operate more efficiently.

From a customer perspective, LLMs provide significant benefits in understanding user sentiment and feedback. Marketing teams, for instance, can analyze the emotions conveyed in user feedback, allowing for the development of more targeted and emotionally resonant marketing strategies. By leveraging this nuanced understanding of user sentiment, brands can create campaigns that go beyond mere communication, fostering deeper connections with their audiences. This emotional intelligence enables marketers to anticipate user needs and preferences, making their messaging more compelling and effective. Similarly, the role of LLMs is crucial in enhancing user experience (UX) design. By mining user feedback for specific suggestions regarding interface improvements, UX designers can pinpoint actionable insights that directly inform design changes. This process ensures that the product interface evolves to meet user expectations more precisely, resulting in a more intuitive and engaging experience. The ability to directly incorporate user feedback into design decisions allows companies to stay responsive and relevant, continually refining their products based on real-world user interactions.

When considering product development, LLMs provide essential tools for organizing and prioritizing feedback. Developers can analyze user input to identify the most frequent issues and highly requested features, allowing them to focus their efforts where they are needed most. This feedback-driven approach to development ensures that user needs are at the forefront of the product evolution process, leading to quicker iterations and more impactful updates. By helping to prioritize tasks based on user data, LLMs contribute to a more streamlined development process, improving both product quality and user satisfaction.

From a product management standpoint, LLMs offer an efficient means of synthesizing large volumes of demo feedback and user interactions into concise, actionable insights. This enables product managers to make data-driven decisions about the product roadmap, ensuring that future updates and features are aligned with user expectations and market trends. As a result, the development process becomes more strategic, with product evolution closely tracking customer demands and preferences. Quality assurance (QA) teams also benefit greatly from the application of LLMs. By detecting patterns and anomalies in feedback related to bugs or performance issues, QA professionals can focus their resources on the most critical problems. This targeted approach enhances the efficiency of testing efforts, leading to more reliable and robust products. The ability to proactively identify and address issues before they escalate ensures higher product quality, which in turn leads to greater customer satisfaction and reduced support overhead.

In conclusion, the adoption of LLMs across customer-facing and product development roles demonstrates their substantial value in enhancing both business operations and user experiences. By facilitating a deeper understanding of user feedback and enabling more informed decision-making, these models help organizations stay agile and responsive to evolving customer needs. As LLM technology continues to advance, its role in driving product innovation and improving customer satisfaction will only grow, solidifying its place as an indispensable tool in the tech industry.


\section{Definition of MVP}

The product goal for the feedback system will focus on simplicity, ease of use, and efficient integration of LLM analysis to provide actionable insights. The system will consist of the following core modules, designed to deliver a smooth feedback collection and analysis experience.

A simple, user-friendly interface for collecting feedback directly from users after they have interacted with the demo. This interface could enable users to share their thoughts directly after interacting with the product. This feedback can be collected via embedded forms or follow-up emails. Once feedback is gathered, a pre-trained LLM processes the data, analyzing natural language inputs to identify key themes, common issues, suggestions, and sentiments. This allows for a deeper understanding of user experience beyond surface-level comments. The results are then displayed on a feedback analysis dashboard, which offers developers, designers, and product managers actionable insights. The dashboard highlights frequent issues, recurring themes, and suggested improvements, providing a clear overview for decision-making. To ensure timely responses to critical feedback, a reporting and notification system is in place. This system automatically alerts relevant team members about urgent feedback or emerging trends, integrating with email or existing project management tools.Given the sensitive nature of user feedback, the system includes basic security measure to safeguard data and ensure compliance with privacy regulations, ensuring that user information is handled responsibly. The system’s automated notifications and strong security measures support timely decision-making and safeguard user data, providing a reliable foundation for continuous product improvement. 

\section{Extensive literature review}



In their groundbreaking paper, "LLM4PLC: Harnessing Large Language Models for Verifiable Programming of PLCs in Industrial Control Systems"\cite{fakih2024llm4plc} delve into the innovative application of Large Language Models (LLMs) for programming Programmable Logic Controllers (PLCs) within industrial settings, a domain that demands high reliability, accuracy, and safety. The research introduces the LLM4PLC framework, an advanced system designed to harness the capabilities of LLMs like GPT-4 and Llama models, which are traditionally known for their robust performance in natural language processing and text generation tasks. This adaptation addresses the critical challenges of automating PLC programming, a field that has traditionally relied heavily on manual coding and extensive verification to meet stringent industrial standards.

PLCs play a pivotal role in the automation landscape, controlling processes that range from manufacturing to critical infrastructure management. The conventional approach to PLC programming involves rigorous testing and validation processes to ensure that the systems operate within safe parameters. However, these methods are not only time-consuming but also susceptible to human error, posing limitations in scenarios that require rapid deployment or modifications.

The introduction of the LLM4PLC framework marks a significant departure from traditional methods. It leverages an iterative, feedback-driven approach that integrates user input and external verification tools directly into the code generation process. This integration allows the system to iteratively refine and verify the generated code, enhancing the reliability and safety of the automated outputs. By incorporating grammar checkers, compilers, and model verifiers like SMV, the framework ensures that the generated PLC code not only meets the functional requirements but also adheres to the critical safety standards necessary for industrial applications.

One of the standout features of the LLM4PLC approach is its ability to significantly reduce the time and expertise required to program PLCs. Traditional programming methods, while reliable, lack the scalability and adaptability that LLM4PLC introduces. The framework’s iterative feedback loop facilitates continuous improvement in code quality, which is instrumental in environments where system requirements can evolve rapidly.

Furthermore, the paper provides a comprehensive comparative analysis illustrating how LLM4PLC enhances the efficiency of the PLC programming process compared to conventional methods. This comparison not only highlights the potential time and cost benefits but also underscores the potential of LLM-based frameworks to revolutionize the field of industrial automation.


As detailed in the second paper, titled ‘An LLM-based Vision and Language Cobot Navigation Approach for Human-centric Smart Manufacturing,’\cite{wang2024llm} the manufacturing industry is advancing towards Industry 5.0. This transition marks a paradigm shift aimed at enhancing the synergy between humans and machines, thereby fostering the creation of more adaptable, intelligent, and efficient manufacturing environments.This transformation is deeply rooted in the concept of Human-centric Smart Manufacturing (HSM), which prioritizes the well-being and needs of human operators. A significant contribution to this field is the innovative integration of Large Language Models (LLMs) into the cobot (collaborative robot) navigation systems, which Tian Wang, Junming Fan, and Pai Zheng explore in their study published in the Journal of Manufacturing Systems.

The research revolves around the development of a cobot navigation system that utilizes both vision and language processing to aid human operators in a manufacturing setting. The introduction of LLMs into the navigation system of cobots represents a substantial leap forward in making these systems more responsive and intuitive for human interaction. LLMs, known for their robust few-shot reasoning and generalization capabilities, are leveraged to process natural language commands, which allow for more flexible and natural human-robot interactions. This integration addresses one of the key challenges in contemporary manufacturing setups—the frequent interruptions caused by the need for operators to fetch tools and materials.

The authors detail a sophisticated framework that begins with the 3D reconstruction and annotation of a real Human-Robot Collaboration (HRC) manufacturing scene using point cloud techniques. This digital twin of the manufacturing environment serves as the foundational layer upon which the LLM operates. By understanding natural language commands, the LLM generates Python code that directs an Automated Guided Vehicle (AGV) to navigate the manufacturing floor autonomously, fetching tools as directed by human operators.

A critical aspect of the study is the Pathfinder algorithm used for path planning, which ensures that the AGV navigates efficiently and safely around the manufacturing space. The system is tested using the AI Habitat simulator, demonstrating the AGV’s capability to understand complex instructions and execute tasks effectively, thereby reducing the time operators spend retrieving items and minimizing workflow disruptions.

The empirical validation provided through case studies highlights the system’s ability to comprehend and execute multi-object navigation tasks from complex language instructions, showcasing the potential of LLMs to significantly enhance operational efficiency in smart manufacturing settings. These findings suggest that the integration of LLMs into cobot systems could be a game-changer, enabling more dynamic and flexible manufacturing processes that truly center around the needs and well-being of human workers.

While the study marks a significant advancement in the application of artificial intelligence in manufacturing, the authors acknowledge limitations, such as the manual annotation of the scene. Future research directions include automating this process through semantic segmentation algorithms to further refine the system’s efficiency and applicability in real-world settings.



In the third paper, “Leveraging Error-Assisted Fine-Tuning Large Language Models for Manufacturing Excellence,”\cite{xia2024leveraging} Xia and colleagues highlight the evolving interface between sophisticated language models and industrial manufacturing processes. As the industry progresses towards more integrated and intelligent systems, the adoption of technologies like Large Language Models (LLMs) capable of understanding and generating domain-specific content has become pivotal. The paper discusses how traditional LLMs, such as those used for generic applications, fall short when applied directly to specialized fields like manufacturing due to their lack of domain-specific training and contextual understanding.

The authors propose an innovative approach to bridge this gap by employing error-assisted fine-tuning of LLMs, particularly tailored for the manufacturing sector. This process involves initial fine-tuning using a corpus specifically curated from manufacturing texts, which helps the models learn the nuances and specific terminology of the field. The introduction of error-assisted iterative prompting further refines the model’s ability by dynamically adjusting to the syntactic and semantic needs of manufacturing-related coding tasks.

Significantly, this method not only enhances the model’s ability to generate functionally and syntactically correct code but also ensures that such code aligns with the practical constraints and requirements of manufacturing applications. For example, the LLM is trained to comprehend and execute commands related to specific manufacturing operations, such as CNC machine programming or robotic path planning, with greater accuracy and relevance.
The paper also outlines a closed-loop refinement process that uses feedback from the model’s output to continuously improve its performance. This approach ensures that the model learns from its errors in real-time, adapting its responses to better meet user expectations and the evolving demands of the manufacturing environment.

The potential of this enhanced LLM application is vast, ranging from improving operational efficiencies to reducing the time and resources spent on manual programming of manufacturing tasks. The authors provide evidence of the model’s improved performance through case studies that demonstrate its enhanced ability to handle complex queries and generate reliable code compared to traditional LLM applications.









% trigger a \newpage just before the given reference
% number - used to balance the columns on the last page
% adjust value as needed - may need to be readjusted if
% the document is modified later
%\IEEEtriggeratref{8}
% The "triggered" command can be changed if desired:
%\IEEEtriggercmd{\enlargethispage{-5in}}

% references section

% can use a bibliography generated by BibTeX as a .bbl file
% BibTeX documentation can be easily obtained at:
% http://mirror.ctan.org/biblio/bibtex/contrib/doc/
% The IEEEtran BibTeX style support page is at:
% http://www.michaelshell.org/tex/ieeetran/bibtex/
%\bibliographystyle{IEEEtran}
% argument is your BibTeX string definitions and bibliography database(s)
%\bibliography{IEEEabrv,../bib/paper}
%
% <OR> manually copy in the resultant .bbl file
% set second argument of \begin to the number of references
% (used to reserve space for the reference number labels box)


% biography section
% 
% If you have an EPS/PDF photo (graphicx package needed) extra braces are
% needed around the contents of the optional argument to biography to prevent
% the LaTeX parser from getting confused when it sees the complicated
% \includegraphics command within an optional argument. (You could create
% your own custom macro containing the \includegraphics command to make things
% simpler here.)
%\begin{IEEEbiography}[{\includegraphics[width=1in,height=1.25in,clip,keepaspectratio]{mshell}}]{Michael Shell}
% or if you just want to reserve a space for a photo:

%\newpage


\bibliographystyle{IEEEtran}
\bibliography{ref}

% You can push biographies down or up by placing
% a \vfill before or after them. The appropriate
% use of \vfill depends on what kind of text is
% on the last page and whether or not the columns
% are being equalized.

%\vfill

% Can be used to pull up biographies so that the bottom of the last one
% is flush with the other column.
%\enlargethispage{-5in}



% that's all folks
\end{document}


